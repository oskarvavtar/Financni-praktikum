\documentclass[11pt]{article}
\usepackage[utf8]{inputenc}
\usepackage[slovene]{babel}

\usepackage{amsthm}
\usepackage{amsmath, amssymb, amsfonts}
\usepackage{relsize}
\usepackage{mathrsfs}

\newcommand{\R}{\mathbb{R}}
\newcommand{\N}{\mathbb{N}}
\renewcommand{\P}{\mathbb{P}}
\newcommand{\E}{\mathbb{E}}
\newcommand{\F}{\mathcal{F}}
\newcommand{\diff}{\overset{\text{def}}{\iff}}
\newcommand{\set}[1]{\{#1\}}

\theoremstyle{definition}
\newtheorem{definicija}{Definicija}[section]

\theoremstyle{definition}
\newtheorem{trditev}{Trditev}[section]

\theoremstyle{definition}
\newtheorem{izrek}{Izrek}[section]

\theoremstyle{definition}
\newtheorem{metoda}{Metoda}[section]

\newtheorem*{posledica}{Posledica}
\newtheorem*{opomba}{Opomba}
\newtheorem*{komentar}{Komentar}
\newtheorem{lema}{Lema}
\newtheorem*{notacija}{Notacija}
\newtheorem*{dokaz}{Dokaz}
\newtheorem*{posplošitev}{Posplošitev}
\newtheorem*{dogovor}{Dogovor}
\newtheorem*{sklep}{Sklep}

\title{Finančni praktikum - definicije, trditve in izreki}
\author{Oskar Vavtar \\
po predavanjih profesorja Janeza Bernika}
\date{2021/22}

\begin{document}
\maketitle
\pagebreak
\tableofcontents
\pagebreak

% #################################################################################################

\section{Uvodna predavanja}
\vspace{0.5cm}

% *************************************************************************************************

\subsection*{6. oktober}
\vspace{0.5cm}

\begin{definicija}

Z obratno indukcijo definiramo 
\begin{align*}
U_T ~&=~ V_T,\\
U_t ~&=~ \max \set{V_t,\max_{\tau \in S_{t,T}} \E(V_t \mid \F_t)}.
\end{align*}
Slučajni proces $(U_t)_{t=0}^T$ imenujemo \textit{Snellova ovojnica} za $(V_t)_{t=0}^T$.

\end{definicija}
\vspace{0.5cm}

% *************************************************************************************************

\subsection*{7. oktober}
\vspace{0.5cm}

\begin{definicija}

$X_n \xrightarrow{(d)} X$ $\diff$ za vsako zvezno omejeno funkcijo \hbox{$f: R^m \rightarrow \R$} velja
$$\E[f(X_n)] ~\longrightarrow~ \E[f(X)].$$
Za $m=1$ je ekvivalentno 
$$F_{X_n}(x) ~\longrightarrow~ F_X(x),$$
za $\forall x$, kjer je $F_X$ zvezna.

\end{definicija}
\vspace{0.5cm}

\begin{trditev}

Naj bo $X_n \sim \text{Bin}(n,p_n)$. Če $\exists \lim_{n \rightarrow \infty} n \cdot p_n = \lambda > 0$, potem zaporedje limitira v porazdelitvi k $\text{Poi}(\lambda)$:
$$X_n ~\xrightarrow[n \rightarrow \infty]{(d)}~ \text{Poi}(\lambda).$$

\begin{proof}
Ideja: dokazati moramo 
$$\P(X_n = k) ~\xrightarrow[n \rightarrow \infty]~ \P(\text{Poi}(\lambda) = k).$$
\end{proof}

\end{trditev}
\vspace{0.5cm}

\begin{definicija}

\textit{Karakteristično funkcijo} slučajne spremenljivke $X$ definiramo kot
$$\varphi_{X_n}(t) ~=~ \E[e^{itX}].$$

\end{definicija}
\vspace{0.5cm}

\begin{izrek}[L\'evyjev izrek]

$X_n \xrightarrow{(d)} X$ $\iff$ $\varphi_{X_n}(t) \xrightarrow[n \rightarrow \infty]~ \varphi_X(t)$ po točkah.

\end{izrek}
\vspace{0.5cm}

% *************************************************************************************************

\pagebreak

% #################################################################################################

\end{document}